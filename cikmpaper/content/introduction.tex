\section{Introduction}

In in a variety of situations from, literature surveys to legal document collection, people try to organize and explore large amounts of documents. Current technology to search on documents are done based on keywords with minor extensions. Sometimes in order to enable keyword search, external effort has to be put in to tag the documents with relevant keywords. Keyword search is useful when the user knows exactly what he or she is looking for. It is not particularly useful when a user wants to explore or learn a new topic. This is especially important when for example, a researcher wants to find out the state of the art in a particular area. In such situations, topic based search can return more relevant results. Topic based search means given a topic, the most relevant content will be shown. In order to accomplish this, the topics among the documents need to be extracted and then they have to be ranked based on relevance to a particular topic. In our system called \emph{Grisham} we present various techniques for topic based exploration and search of peer reviewed scientific papers. Our work allows user to search for scientific papers using three methods - first, users may perform a traditional search keyword based search for papers. Second, users can specify a topic or topics he or she is interested in and the most relevant papers for that topic will be listed in the order of its relevance by our system, and lastly, users are allowed to explore similar or related document using a visual graph like interface, i.e., given a paper, a user will be shown a set of papers which cite the original paper in the order of its relevance to the topic. 


The main contributions of our work are the \emph{user topic ranking function} which ranks the relevance of documents to a set of topics, the \emph{similar document explorations} which is performed by computing the similarity among the papers, and \emph{topic-document visualization} which takes into account the citations of papers while computing topic similarity and ranking.

