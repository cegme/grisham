\section{Introduction}

In in a variety of situations from, literature surveys to legal document
collection, people try to organize and explore large amounts of documents.
Current searches on documents are done based on keywords with minor
extensions. In order to enable keyword search, external effort has to be put in
to tag the documents with relevant keywords. 
Keyword search is useful when the user knows exactly what he or she is looking
for. It does not help when a user wants to explore or learn a new topic. This
is especially important when a researcher wants to find out the state of the
art in a particular area. In such situations, topic based search can return more relevant results. In our
demo we present various ways of topic based exploration and search of
documents. First the topics of the corpus documents are extracted and ranked.
Subsequently, a user can specify the topics he or she is interested in with a
priority criteria and based on the priority, our system lists the most relevant
documents. The main contributions of our work are the \emph{user topic ranking
function}, the \emph{similar document explorations}, and \emph{topic-document
visualization}.


Our main contributions are:
\begin{itemize}
\item User topic ranking functions.
\item Similar document explorations
\item Topic-Document visualization.
\end{itemize}
