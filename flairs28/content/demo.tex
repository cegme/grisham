
\section{User Interface}
\label{sec:demo}

\ceg{Note this section will be spread out into the other sections.}

To demonstrate \system's exploratory search we loaded scientific paper from the DBLP conference~\cite{Tang:2008:EMA:1367497.1367722}.
All the searches in the system require the keep in account the user model.
The user model is a profile of preference provided by the user before she makes any search.
This is done by specifying weightage to various topics.
The {\system} website has a list of topics with a slider associated with each of them using which a user can specify the degree of interest in that particular topic.
This array of user preference is used as the ranking factor in all the results.
The website has three basic functionality which has been classified under three different tabs of the same name.
They are keyword paper search, Topic explore, and Graph explore. 

\paragraph{Keyword Paper Explore}
This is a universal search facility.
The user may enter one or more keywords representing topics, or author names etc.
The keywords are searched in the title, abstract, and the author names of all the papers are listed.
The listing is ranked based on the user model.
In the first page only a few of the papers are displayed in two boxes --- one containing matching words in the title, and the other in the abstract.
Clicking on any box will open up a more complete list of papers.

\begin{figure}[htb]
\includegraphics[width=.5\textwidth]{images/topic_exploration.png} % scale=.25,trim=0 0 300 0
\caption{User configured topic exploration}
\label{fig:topic_exploration}
\end{figure}

\paragraph{Topic Explore} The second tab on the website is for topic exploration and it allows a user to click on a specific topic to know more about the papers associated with the topic.
Initially, all the extracted topics from the corpus are shown along with their topic words.
This list is color coded to distinguish the relevance of topics as indicated by the user model --- much like a heat map.
The list is clickable and one a topic is clicked, relevant papers to that topic ranked based on the user model and displayed.
The user can look at this heat map to adjust their topic preferences.
We use equation~\ref{eq:KL} on the client side to calculate this preference. 
In the graph explorer the citations for the current paper is ranked
using equation~\ref{eq:KL}. The citations of that paper that are most
similar to the user model are ranked the highest.
A screenshot is shown in Figure~\ref{fig:topic_exploration}.





\begin{figure}[htb]\centering 
\includegraphics[width=.5\textwidth]{images/topical_docs.png}
\caption{Visualizing of Topic-Based Search in \system, orange 
circles represent documents and blue circles represent the estimated 
topics in a corpus.}
\label{fig:topic-search-viz}
\end{figure}

\begin{figure*}[htb]\centering 
\includegraphics[width=1\textwidth]{images/para_topic_distribution.png}
\caption{Visualizing the topic distribution of the introduction 
section of the Wikipedia article \textit{Killer Whale}. See Figure~\ref{fig:doc-topic-distribution} for the topic distribution of 
the whole article.}
\label{fig:doc-word-counts}
\end{figure*}

\begin{figure}[htb]\centering 
\includegraphics[width=.5\textwidth]{images/doc_tf_bubble_chart.png}
\caption{Visualizing the relative frequencies of terms in the 
Wikipedia article \textit{Killer Whale}. \cpg{I think we need to 
decide whether we should keep this figure in the paper or not.}}
\label{fig:doc-word-counts}
\end{figure}

\paragraph{Graph Explore}
One of the most interesting visualizations in the project is the graph explore which allows a user to conceptually drill down a graph of a paper and its citations in a recursive sequence.
This is a common method of literature exploration; a user takes up a base paper and then reads up all the papers which have been cited in that base paper.
These steps are recursively performed for each subsequent paper until the user has found a sufficient amount of papers or they read all the papers in their collection.
%Though effective, this technique tell the user which ones he should pursue and which ones he should now.

The graph explore functionality allows a user to perform this in a more visually appealing.
Once a user decides on a base paper (through keyboard search), the system shows a graph representation of its citations which are ranked based on her profile (user model).
This will help her pursue the most relevant papers first.
Clicking on any secondary paper will open up the graph further and list it's citations ranked taking into account the user model.
