\section{Introduction}

In a variety of situations, from literature surveys to legal 
document collections, people try to organize and explore large 
amounts of documents. Current technology to search on documents are 
done based on keywords with minor extensions. Keyword-based search is useful when the 
user knows exactly what he or she is looking for. It is not 
particularly useful when a user wants to explore or learn a new 
topic. 
The difficulty with keyword based search is especially pronounced when, for
example, a researcher wants to find out the state of the art in a particular
area or a student would like to create a literature survey. In such situations,
\textsl{topic-based search} can return more relevant results. 
Topic-based search is a classification of the search space to 
highlight topical relevance. In order to accomplish this, the topics 
underlying the document collection need to be extracted, and then 
they have to be represented in terms of those topics and ranked 
based on relevance to a particular topic. 

In our system called \system we present various techniques for topic-based exploration 
and search of articles. Our work allows user 
to search for documents using three methods: 
First, users may perform a traditional keyword-based search for papers.
The results are re-ranked based on a topic distribution specified by the user. 
Second, user may perform a \textsl{topic-based search} where the user specifies 
a topic she is interested in and documents will be returned ordered 
by their relevancy to the topic.
Lastly, users 
may explore topically similar documents using a visual 
graph like interface. In all 
three methods, we adapt search results to personalize results.

\eat{The main contributions of our work are the \textit{user-topic ranking function} 
which ranks the relevance of documents to a set of topics---\cpg{I 
think we need to edit this; a similar ranking function is there in 
\citeauthor{George2012}~\citeyear{George2012}}, the 
\textit{similar document explorations} which is performed by computing the 
similarity of papers to a topic of interest, and \textit{topic-document visualization} which 
ranks citations for a paper by the interest of the user.
}

This work is an implementation of a system that uses a novel search and 
exploration paradigm. The main contributions of this work are as follows:
\begin{itemize}
\item A search based on user-defined preferences. A user can
interactively define a topic-based model for their search preferences.
\item An implementation of ranking functions for topic-based search.
\item A novel interaction model for topic-based exploration.
\end{itemize}

This paper is organized as follows. First, we give an introduction 
to topic models. Next, we describe the theoretical structure of \system.
We then describe the user interface and the user interaction model.
Finally, we conclude this paper with a discussion of related work.





