\section{Introduction}

In a variety of situations, from literature surveys to legal 
document collections, people try to organize and explore large 
amounts of documents. Current technology to search on documents are 
done based on keywords with minor extensions. Sometimes in order to 
enable keyword search, external effort---\cpg{What are they?}--- has to be put in to tag the 
documents with relevant keywords. Keyword-based search is useful when the 
user knows exactly what he or she is looking for. It is not 
particularly useful when a user wants to explore or learn a new 
topic. This is especially important when for example, a researcher 
wants to find out the state of the art in a particular area or a 
student would like to create a literature survey. In such situations, 
\textsl{topic-based search} can return more relevant results. 
Topic-based search is a classification of the search space to 
highlight topical relevance. In order to accomplish this, the topics 
underlying the document collection need to be extracted, and then 
they have to be represented in terms of those topics and ranked 
based on relevance to a particular topic. 

In our system called 
\system we present various techniques for topic-based exploration 
and search of peer reviewed scientific papers. Our work allows user 
to search for scientific papers using three methods: First, users 
may perform a traditional search \textsl{keyword-based search} for 
papers. Second, users can specify a topic or topics he or she is 
interested in and the most relevant papers for that topic will be 
listed in the order of their relevance by our system. Lastly, users 
are allowed to explore similar or related documents using a visual 
graph like interface, i.e., given a paper or an article, a user will 
be shown a set of papers which cite the original paper in the order 
of their relevance to the topic. In addition, we allow the user to 
specify a topic distribution that specifies their interest. In all 
three methods, we adapt search results to personalize results.

The main contributions of our work are the \textit{user-topic ranking function} 
which ranks the relevance of documents to a set of topics---\cpg{I 
think we need to edit this; a similar ranking function is there in 
\citeauthor{George2012}~\citeyear{George2012}}, the 
\textit{similar document explorations} which is performed by computing the 
similarity of papers to a topic of interest, and \textit{topic-document visualization} which 
ranks citations for a paper by the interest of the user.
