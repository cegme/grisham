\section{Topic Models}

% In this section we give a brief summary of topic models.
% \ceg{Clint you can fill this in}

Topic modeling allows us to represent the properties of a 
large collection of documents containing numerous words 
with a small collection of topics. 
A topic is represented by a  
distribution of words in a vocabulary.  This approach to 
topic modeling hinges on the bag-of-words model in which 
word occurrences within a document are assumed to be 
independent of each other. 
Latent Dirichlet Allocation (LDA) is a well known, probabilistic 
topic model, which can represent hidden topic structues 
of the documents in a text corpus \cite{Blei2003}. Due to 
its fully generative semantics, even at the level of documents, 
LDA could address the drawbacks of other topic models such as 
the Latent Semantic Analysis (LSA) and Probabilistic LSA \cite{Hofmann1999,Blei2003}. 
In LDA, each document is described by 
a mixture of topics, and words are chosen from the multinomial 
that results from the mixture of the documents topic 
multinomials. The topics themselves, as well as the 
documents are drawn from Dirichlet distributions. 

Topic model analysis depends upon exploring the posterior 
distribution of model parameters and hidden variables 
conditioned on observed words. The model parameters are 
corpus-level topics and document-level topic mixtures. 
The topic models such as LSA and LDA assumes that 
we know a-priori the number of topics in the corpus. 
People have used different approximate inference methods
such as deterministic approaches -- replace the posterior 
integral with a tractable lower bound \cite{Blei2003}, 
and sampling methods -- approximate the posterior integral 
using an empirical average \cite{Griffiths2004}, for the 
inference with LDA. 


