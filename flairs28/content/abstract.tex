
\begin{abstract}

\eat{We describe a demonstration of our system \system, presenting 
methods for topic-based paper exploration. We use a popular topic 
modeling algorithm, Latent Dirichlet Allocation, to derive topic 
distributions for each scientific paper in a DBLP citation data set. 
We allow users to specify personal topic distribution to 
contextualize the exploration experience. We demonstrate three types 
of exploration \textsl{keyword-based search}, \textsl{topic-based 
exploration}, and \textsl{citation-lineage search}. In each model we 
shape the results to be specific to a user specification. We 
describe the components of our web-based system.}


From literature surveys to legal 
document collections, people try to organize and explore large 
amounts of documents.
During these tasks, students and researchers commonly search for documents based on particular themes.
In this paper we use a popular topic modeling algorithm, Latent Dirichlet Allocation, to derive topic distributions for articles.
We allow users to specify personal topic distribution to 
contextualize the exploration experience.
We introduce three types 
of exploration: \textsl{user model re-weighted keyword search},
\textsl{topic-based search} and \textsl{topic-based exploration}. 
We demonstrate these methods using the a scientific citation 
data set and a Wikipedia article collection.
We also describe the user interaction model.

\end{abstract}
