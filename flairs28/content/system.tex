\section{\system}

%\ceg{This section will be split into \textbf{SEARCH} and \textbf{EXPLORATION}}

\ceg{STILL NEED TO REARRANGE PARAGRAPSH AND REWORD}
\system is a web-based system build on top of the PostgreSQL
open source database system. Computation for visualization is performed
using HTML, CSS and javascript.
Calculations are performed using in-database computation through AJAX calls
and client side javascript.
In this section we discuss each part of \system. 

\subsection{Data pre-processing and topic learning}
First, we tokenize the raw text based on the python NLTK toolkit and a
predefined regular expression. Second, we standardize them by removing noise terms
and stop-words. Third, we represent each document in a sparse 
bag-of-words format, after building a vocabulary of
corpus-words. Last, we use them as input to the topic learning model which will
in turn learn clusters from the term co-occurrence frequencies of the corresponding
documents. In this project, we used the Gensim 
package implementation of the LDA online learning algorithm 
\cite{rehurek_lrec,hoffman2010online} for topic learning, which is 
based on the variational inference framework.    

Components of a learned topic model includes the corpus-level topic word association counts
and document-level topic mixtures. Each estimated topic is represented
by its topic-word-counts, which is useful for our automatic detection of paper topics. 
The document-level topic mixtures give an idea of the topicality of a
particular paper given a topic. This is also quite useful in finding similar papers and
grouping them together, because topic modeling is a type of dimensionality reduction
technique that enables us to work on the topic-space rather than on 
the vocabulary-space.






\subsection{User Model}
When performing search, exploration and discovery over academic papers users 
may bring particular context to their search. Incorporating this information
into the search process has been show to be beneficial to 
users~\cite{DZSRWJ,MZPGSOL}.
We develop a user model that 
encapsulates the users personal context and integrates it into their
search task. 

This model is a distribution of weights for each topic.
Formally, given a set of topics $T$ the user model is defined as
$$
\mathcal{U} = \{u_0, \ldots, u_{|T|}\}
$$
where $u_i \in [0,1]$ and $\sum_{t \in T} u_t = 1$.
We graphically allow the user to select the weights that correspond to
each topic. This allows the users to change preferences with each query
for more desirable results.

The user model is used in different ways to provide better feedback to
the user. After a keyword search, the document results of the search 
are re-ranked by calculating the KL-divergence of each document and the
user model. Formally, given the set of result documents $D$:

\begin{equation} \label{eq:KL}
KL(\mathcal{U}||d) = \sum_{t \in T} u_t \ln \frac{u_t}{d_t}.
\end{equation}
where $d \in D$ and $d_t$ is the topic proportion for document $d$ and
topic $t \in T$. 

In the topic explorer, each topic row is color-coded like a heat 
based map based on the similarity of the user model to that topic (see Figure~\ref{fig:topic_exploration}).
The user can look at this heat map to adjust their topic preferences.
We use equation~\ref{eq:KL} on the client side to calculate this preference. 
In the graph explorer the citations for the current paper is ranked
using equation~\ref{eq:KL}. The citations of that paper that are most
similar to the user model are ranked the highest.


\subsection{Topic-Based Search}

\subsubsection{Ranking Function}
We provided several ranking functions to let the user find
the best papers.

One way of ranking is to determine relevant papers 
given an estimated topic. We use individual papers'
document topic mixtures, $\hat{\theta_d}$, to rank them 
on relevance given a topic. For a given 
topic $t \in K$, we calculate
\begin{equation}
m(d) = \ln \hat{\theta}_{d,t} + \sum_{j \neq t}{\ln (1 - \hat{\theta}_{d,j})}
\end{equation}
for all documents $d = 1, 2, \ldots, D$ in 
the document collection, and sort them to rank
them on relevance. Here, we assume that
each $\hat{\theta}_d$ is normalized, 
i.e., $\sum_{j=1}^{K}{\hat{\theta}_j} = 1$.
Intuitively, we can see that this equation maximizes the probability
of a topic $t \in K$ given a document.
That means a document with a higher value of this score is 
highly relevant for the selected topic $t$, and contains 
a considerable amount of words from the topic distribution.    



\subsection{Topic-Based Exploration}


\ceg{clint stuff}

\subsubsection{Lineage search}


%\subsection{Visualization}
%At times it is important to allow the user to do exploritory search.

