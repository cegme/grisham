
\section{Discussion and Related Work}

There have been previous systems to use topic modeling as a basis of
search and exploration. 
%We are the first to our knowledge to incorporate
%a user modeling into the loop.

A system of note is Yang et al.~\cite{yang2011topic} who applied topic modeling to 
collections of historical news papers to assist search. They 
found that the topics generated from topic models are 
are generally good, however once the sets of topics are 
generated, an expert opinion is required to name them. 
In \system, we allow users to select numbered topics for article-search 
based on topically relevant words.

%Blog noon \cite{Grineva:2011:BET:1963192.1963292}.


Termite~\cite{2012-termite} provides a visual analytic tool for assessing topic quality that allows for comparison of terms within and across latent topics. They introduce a saliency measure that enables the selection of relevant terms. For a particular topic, the system provides the word frequency distribution relative to the full corpus and shows the most representative terms according to the saliency measure.

A number of previous work~\cite{chang2009reading,mimno2011optimizing,newman2010evaluating} depended heavily on experts examining lists of the most probable words in the topic and validating the models. Hall et al.~\cite{hall2008studying} applied unsupervised topic modeling to study historical trends in computational linguistics across 14,000 publications. The work required experts  to validate the quality of the results. Only 36 out of 100 topics were retained, and there were 10 additional topics that were not produced by the model and had to be manually inserted.

Leake et al~\cite{leake2003topic} provide methods to aid concept mapping by suggesting relevant information in the context of topics models, represented as concept maps. 

For visualizing the results, previous work~\cite{2012-termite,bertin1983semiology,henry2007matlink} use
 a matrix style view to surface the relationships between many terms. 
%However, it requires an appropriate ordering.
These tools are created for evaluating topic models.
Interacting with such visualizations can be complex because the user should
already have an intuition about the results in advance in order to properly generate necessary
orderings.






